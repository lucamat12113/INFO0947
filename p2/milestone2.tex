\documentclass[a4paper, 11pt, oneside]{article}

\usepackage[utf8]{inputenc}
\usepackage[T1]{fontenc}
\usepackage[french]{babel}
\usepackage{array}
\usepackage{shortvrb}
\usepackage{listings}
\usepackage[fleqn]{amsmath}
\usepackage{amsfonts}
\usepackage{fullpage}
\usepackage{enumerate}
\usepackage{graphicx}             % import, scale, and rotate graphics
\usepackage{subfigure}            % group figures
\usepackage{alltt}
\usepackage{url}
\usepackage{indentfirst}
\usepackage{eurosym}
\usepackage{listings}
\usepackage{color}
\usepackage[table,xcdraw,dvipsnames]{xcolor}


% Change le nom par défaut des listing
\renewcommand{\lstlistingname}{Extrait de Code}

% Change la police des titres pour convenir à votre seul lecteur
\usepackage{sectsty}
\allsectionsfont{\sffamily\mdseries\upshape}
% Idem pour la table des matière.
\usepackage[nottoc,notlof,notlot]{tocbibind}
\usepackage[titles,subfigure]{tocloft}
\renewcommand{\cftsecfont}{\rmfamily\mdseries\upshape}
\renewcommand{\cftsecpagefont}{\rmfamily\mdseries\upshape}

\definecolor{mygray}{rgb}{0.5,0.5,0.5}
\newcommand{\coms}[1]{\textcolor{MidnightBlue}{#1}}

\lstset{
    language=C, % Utilisation du langage C
    commentstyle={\color{MidnightBlue}}, % Couleur des commentaires
    frame=single, % Entoure le code d'un joli cadre
    rulecolor=\color{black}, % Couleur de la ligne qui forme le cadre
    stringstyle=\color{RawSienna}, % Couleur des chaines de caractères
    numbers=left, % Ajoute une numérotation des lignes à gauche
    numbersep=5pt, % Distance entre les numérots de lignes et le code
    numberstyle=\tiny\color{mygray}, % Couleur des numéros de lignes
    basicstyle=\tt\footnotesize,
    tabsize=3, % Largeur des tabulations par défaut
    keywordstyle=\tt\bf\footnotesize\color{Sepia}, % Style des mots-clés
    extendedchars=true,
    captionpos=b, % sets the caption-position to bottom
    texcl=true, % Commentaires sur une ligne interprétés en Latex
    showstringspaces=false, % Ne montre pas les espace dans les chaines de caractères
    escapeinside={(>}{<)}, % Permet de mettre du latex entre des <( et )>.
    inputencoding=utf8,
    literate=
  {á}{{\'a}}1 {é}{{\'e}}1 {í}{{\'i}}1 {ó}{{\'o}}1 {ú}{{\'u}}1
  {Á}{{\'A}}1 {É}{{\'E}}1 {Í}{{\'I}}1 {Ó}{{\'O}}1 {Ú}{{\'U}}1
  {à}{{\`a}}1 {è}{{\`e}}1 {ì}{{\`i}}1 {ò}{{\`o}}1 {ù}{{\`u}}1
  {À}{{\`A}}1 {È}{{\`E}}1 {Ì}{{\`I}}1 {Ò}{{\`O}}1 {Ù}{{\`U}}1
  {ä}{{\"a}}1 {ë}{{\"e}}1 {ï}{{\"i}}1 {ö}{{\"o}}1 {ü}{{\"u}}1
  {Ä}{{\"A}}1 {Ë}{{\"E}}1 {Ï}{{\"I}}1 {Ö}{{\"O}}1 {Ü}{{\"U}}1
  {â}{{\^a}}1 {ê}{{\^e}}1 {î}{{\^i}}1 {ô}{{\^o}}1 {û}{{\^u}}1
  {Â}{{\^A}}1 {Ê}{{\^E}}1 {Î}{{\^I}}1 {Ô}{{\^O}}1 {Û}{{\^U}}1
  {œ}{{\oe}}1 {Œ}{{\OE}}1 {æ}{{\ae}}1 {Æ}{{\AE}}1 {ß}{{\ss}}1
  {ű}{{\H{u}}}1 {Ű}{{\H{U}}}1 {ő}{{\H{o}}}1 {Ő}{{\H{O}}}1
  {ç}{{\c c}}1 {Ç}{{\c C}}1 {ø}{{\o}}1 {å}{{\r a}}1 {Å}{{\r A}}1
  {€}{{\euro}}1 {£}{{\pounds}}1 {«}{{\guillemotleft}}1
  {»}{{\guillemotright}}1 {ñ}{{\~n}}1 {Ñ}{{\~N}}1 {¿}{{?`}}1
}
\newcommand{\tablemat}{~}

%%%%%%%%%%%%%%%%% TITRE %%%%%%%%%%%%%%%%
% Complétez et décommentez les définitions de macros suivantes :
\newcommand{\intitule}{Polylignes, Milestone 1}
\newcommand{\GrNbr}{06}
\newcommand{\PrenomUN}{Maxime}
\newcommand{\NomUN}{Deravet}
\newcommand{\PrenomDEUX}{Luca}
\newcommand{\NomDEUX}{Matagne}
% Décommentez ceci si vous voulez une table des matières :
\renewcommand{\tablemat}{\tableofcontents}

%%%%%%%% ZONE PROTÉGÉE : MODIFIEZ UNE DES DIX PROCHAINES %%%%%%%%
%%%%%%%%            LIGNES POUR PERDRE 2 PTS.            %%%%%%%%
\title{INFO0947: \intitule}
\author{Groupe \GrNbr : \PrenomUN~\textsc{\NomUN}, \PrenomDEUX~\textsc{\NomDEUX}}
\date{}
\begin{document}

\maketitle
\newpage
\tablemat
\newpage
%%%%%%%%%%%%%%%%%%%% FIN DE LA ZONE PROTÉGÉE %%%%%%%%%%%%%%%%%%%%

%%%%%%%%%%%%%%%% RAPPORT %%%%%%%%%%%%%%%
% Écrivez votre rapport ci-dessous.
\section{Remarques}

Une section complète étant dédiée aux opérations nécessitant un invariant, celles-ci ne se retrouveront donc pas dans la section "structure de données"

\section{Structures de données}
\subsection{Point2D: fonctions et structure}

\begin{lstlisting}
struct Point2D{
	float x;
	float y;
};
\end{lstlisting}

\begin{lstlisting}[escapeinside={(*}{*)}]

/* 
 * @pre: /
 * @post: (get_x(create_Point2D) = x 
 *(* \color{blue}{$\land$} *)
 * get_y(create_Point2D) = y) 
 */
Point2D* CreatePoint2D(float x, float y);
/* 
 * @pre: (*\color{blue}{A != NULL $\land$ B != NULL}*)
 * @post: (*\color{blue}{$A=Translate_{(A, B)} \land B=B_0$}*)
 */
void TranslatePoint2D(Point2D* A, Point2D* B);
/* 
 * @pre: (*\color{blue}{A != NULL $\land$ B != NULL}*)
 * @post: (*\color{blue}{$A=Rotate_{(A, B), x} \land B=B_0 $}*)
 */
void RotatePoint2D(Point2D* A, Point2D* B, float x);
/* 
 * @pre: A != NULL
 * @post: (*$\color{blue}{A=A_0  \land }  get_x = x$ *)
 */
float get_x(Point2D* A);
/* 
 * @pre: A != NULL
 * @post: (*$\color{blue}{A=A_0  \land }  get_y = y$ *)
 */
float get_y(Point2D* A);
/* 
 * @pre: A != NULL (*\color{blue}{$\land $} *) B != NULL
 * @post: (*$ \color{blue}{A=A_0 \land B=B_0 \land  EuclDist = \sqrt{(X_a-X_b)+(Y_a-Y_b)} } $*)
 */
unsigned float EuclDist(Point2D* A, Point2D* B);
\end{lstlisting}

\subsection{Polyligne: fonctions et structure}

\begin{lstlisting}
struct Polyline{
	boolean open;
	unsigned nbpoint;
	unsigned float length;
	unsigned arraySize;
	Point2D** pointArray;
};
\end{lstlisting}

\begin{lstlisting}[escapeinside={(*}{*)}]
/* 
 * @pre: A != NULL (*\color{blue}{$\land$} B != NULL \color{blue}{$\land$}*)
 * @post: (*\color{blue}{$A=A_0$ $\land$ $B=B_0$ $\land$ open=open $\land$ create\_Polyligne = P $\land$ }*)
 * (*\color{blue}{nbpoint(P) = NbrPoint(P) $\land$ length(P) = Length(P)}*)
 */
Polyline* CreatePolyline(Point2D* A, Point2D* B, boolean open);
/* 
 * @pre: P != NULL 
 * @post: (*\color{blue}{$P=P_0$ $\land$ open = False $\land$  $nbpoint = nbpoint_0 $*)
 */
void Open(Polyline* P);
/* 
 * @pre: P != NULL 
 * @post: (*\color{blue}{$P=P_0$ $\land$ open(P) = True $\land$ $nbpoint = nbpoint_0 $ *)
 */
void Close(Polyline* P);
/* 
 * @pre: P != NULL 
 * @post: (*\color{blue}{$P=P_0$ $\land$*)
 */
void IsOpen(Polyline* P);
(*{\color{red} Poser la question close et open}*)
/* 
 * @pre: P != NULL
 * @post: (*\color{blue}{P=P$_0$ $\land$} nbpoint = NbrPoint(P) *)
 */
unsigned NbrPoint(Polyline* P);
/* 
 * @pre: (*\color{blue}{$P != NULL  \land$ } numero < nbpoint*)
 * @post: (*$\color{blue}{P=P_0 \land} GetPoint = {A_{numero}}$*)
 */
Point2D GetPoint(Polyline* P, unsigned numero);
/*
 * @pre: (*\color{blue}{ P != NULL $\land$ A != NULL }*)
 * @post: (*\color{blue}{$A=A_0$ $\land$ $open =open_0$ $\land$} \color{blue}{$nbpoint = nbpoint_0 + 1$  *)
 */
(*{\color{red} Poser la question distance en post condition, recalculer?}*)
void AddPoint2D(Polyline* P, Point2D* A);
/*
 * @pre: (*\color{blue}{ P != NULL $\land$ A != NULL }*)
 * @post: (*\color{blue}{$A=A_0$ $\land$ $open =open_0$ $\land$} \color{blue}{$nbpoint = nbpoint_0 - 1$  *)
 */
 void SuppPoint2D(Polyline* P);
\end{lstlisting}

\section{Invariants}
A)ajouter les tableaux d'invariant et les invariants formels vendredi après le boulot

B)finir rotate et translate
\subsection{length}
\begin{lstlisting}[escapeinside={(*}{*)}]
/* 
 * @pre: P != NULL
 * @post: (*\color{blue}{P=P$_0$ $\land$} length = Length(P) *)
 */
unsigned float length(Polyline* P);
\end{lstlisting}

\subsection{Rotate}


\subsection{Translate}


\end{document}
